\documentclass[a4paper]{article}

%%%%%%%%%%%%%%%%%%%%%%%%%%%%%%%%%%%%%%%%%%%%%%%%%%%%%%%%%%%%%%%%%%%%%%%%%%%%
% Some common includes. Add additional includes you need.
%%%%%%%%%%%%%%%%%%%%%%%%%%%%%%%%%%%%%%%%%%%%%%%%%%%%%%%%%%%%%%%%%%%%%%%%%%%%
\RequirePackage{ngerman}
\RequirePackage[utf8]{inputenc}
\RequirePackage[T1]{fontenc}
\RequirePackage[margin=23mm,bottom=30mm]{geometry}
\RequirePackage{graphicx}
\RequirePackage{amsmath,amsfonts,amssymb,amsthm}
\RequirePackage{hyperref}

%%%%%%%%%%%%%%%%%%%%%%%%%%%%%%%%%%%%%%%%%%%%%%%%%%%%%%%%%%%%%%%%%%%%%%%%%%%%
% Defines for mathematical notation. Add additional defines as needed.
%%%%%%%%%%%%%%%%%%%%%%%%%%%%%%%%%%%%%%%%%%%%%%%%%%%%%%%%%%%%%%%%%%%%%%%%%%%%
\def\O{\mathcal{O}}
\def\sort{\mathrm{sort}}
\def\scan{\mathrm{scan}}
\def\dist{\mathrm{dist}}

\setlength{\parindent}{0cm}
\renewcommand{\refname}{Quellen}
%%%%%%%%%%%%%%%%%%%%%%%%%%%%%%%%%%%%%%%%%%%%%%%%%%%%%%%%%%%%%%%%%%%%%%%%%%%%
% Definition of the assignment header
%%%%%%%%%%%%%%%%%%%%%%%%%%%%%%%%%%%%%%%%%%%%%%%%%%%%%%%%%%%%%%%%%%%%%%%%%%%%
\input{/Users/larspetersen/JWGU/Prg/header.tex}
%%%%%%%%%%%%%%%%%%%%%%%%%%%%%%%%%%%%%%%%%%%%%%%%%%%%%%%%%%%%%%%%%%%%%%%%%%%%

% Set option "german" or "english", depending on what language the
% default texts should be in.
\ExecuteOptions{german}
\ProcessOptions

% Enter the lecture name and semester
\lecture{Einf\"uhrung in die Programmierung}
\semester{Winter 2016/2017}

% Enter your data: Name, Matrikelnummer (student ID number) and group
\student{Qasim Raza, Lars Petersen}{6360278, 6290157}{11}
% Tutorin: sabrinasafre@gmail.com

% Which assignment is this?
\assignment{5}

% The environment "exercise" takes one parameter (the exercise number). 
% This way you can skip exercises if you like. Example:
% 
% \assignment{3}
% \begin{exercise}{8}
% ...
% \end{exercise}
% 
% The solution to exercise 3.8 (3rd assignment, 8th exercise) goes where 
% the dots are.


\begin{document}



\begin{exercise}{x}

Dokumentation der Testf\"alle:


\begin{center}
	\begin{tabular}{| p{2.5cm} | p{2.2cm} | p{10cm} |}
		\hline
		Funktionalit\"at & Eingabe & Verhalten des Programms\\ \hline \hline
		
		Begr\"u\ss{}ung & 
		& In der Konsole wird eine allgemeine Begr\"u\ss{}ung ausgegeben. \\ \hline
		
		Abfrage des Initialisierungsmodus & $0$ - random \newline $1$ - manuelle Eingabe
		& Der Benutzer hat w\"ahrend der Initialisierungsphase die M\"oglichkeit, die
		Parameter des Spiels randomisiert erzeugen zu lassen, oder diese selbst einzugeben.\\ \hline
		
		Startwerte f\"ur die St\"adte & random
		& Aus einer Liste werden zuf\"allig zwischen 5 und 20 St\"adte ausgew\"ahlt. \\ \hline
		
		Startwerte f\"ur die St\"adte & manuell
		& Der Benutzer kann selbstst\"andig eine Liste zwischen 5 und 20 St\"adten eingeben. \\ \hline
		
		Startwerte f\"ur Heimatstadt und Anzahl Manager & random
		& Es wird zuf\"allig eine Heimatstadt aus der Liste der St\"adte ausgew\"ahlt.
		Zudem wird zuf\"allig eine Anzahl von Managern gew\"ahlt (zwischen 5 und 20). \\ \hline
		
		Startwerte f\"ur die St\"adte & manuell
		& Der Benutzer kann selbstst\"andig die Heimatstadt aus der Liste der St\"adte ausw\"ahlen.
		Zudem kann er die Anzahl der dortigen Manager festlegen (zwischen 5 und 20). \\ \hline
		
		Startwerte f\"ur Spieldauer & random
		& Es wird zuf\"allig eine Spieldauer festgelegt (zwischen 5 und 40).
		Zudem wird zuf\"allig eine Anzahl von Managern gew\"ahlt (zwischen 5 und 40). \\ \hline
		
		Startwerte f\"ur die Spieldauer & manuell
		& Der Benutzer kann selbstst\"andig die Spieldauer angeben (zwischen 5 und 40). \\ \hline
		
		Startwerte f\"ur Gewinne in St\"adten & random
		& Es wird zuf\"allig je Stadt der potentielle Gewinn festgelegt (zwischen -20 und 90). \\ \hline
		
		Startwerte f\"ur Gewinne in St\"adten & manuell
		& Der Benutzer kann selbstst\"andig je Stadt den potentiellen Gewinn festlegen (Angabe
		einer Liste mit Werten zwischen -20 und 90). \\ \hline
		
		Startwerte f\"ur Stra\ss{}ennetzwerk & random
		& Es wird zuf\"allig ein Stra\ss{}ennetzwerk initialisiert und angezeigt. \\ \hline
		
		Startwerte f\"ur Stra\ss{}ennetzwerk & manuell
		& Der Benutzer kann selbstst\"andig das Netzwerk festlegen, indem er Adjazenzen \"andert.
		Dabei geht er vom vollst\"andig zusammenh\"angenden Netzwerk aus. Jede einzelne 
		\"Anderung wird ihm angezeigt. Mit end! kann er seine \"Anderungen abschlie\ss{}en. \\ \hline
		
		Anzeige des Stra\ss{}ennetzes &
		& In einer Matrix werden die Verbindungen des Stra\ss{}ennetzes angezeigt. \\ \hline
		
		Anzeige des Spielstatus zu Beginn eines Spieltages &
		& Zu Beginn des Spieltages wird dem Spieler immer der Status des Spiels angezeigt.
		Dies umfasst die Adjazenzmatrix und die Verteilung der Manager, Hotels sowie die
		potentiellen Gewinne. \\ \hline
		
		Ablauf einer Spielrunde &
		& Die Spielrunde beginnt mit der bisherigen Statistik des Spielers w\"ahrend des Spiels,
		darauf folgt der input, woraufhin das Programm diesen Befehl umsetzt. \\ \hline
		
	\end{tabular}

	\begin{tabular}{| p{2.5cm} | p{2.2cm} | p{10cm} |}
		\hline
		Funktionalit\"at & Eingabe & Verhalten des Programms\\ \hline \hline
		
		Spielzug \texttt{pass} & pass
		& Mit diesem Befehl nimmt der Benutzer keine Bewegung oder \"Anderung vor.
		Die Syntax der Eingabe wird gepr\"uft. Nicht korrekte Eingaben ziehen eine neue Eingabe
		nach sich. \\ \hline
		
		Spielzug \texttt{move} & move: number, city1, city2
		& Mit diesem Befehl nimmt der Benutzer keine Bewegung oder \"Anderung vor.
		Die Syntax und die Semantik der Eingabe (St\"adte existieren, St\"adte sind verbunden,
		Anzahl Manager in city1 ist gr\"o\ss{}er gleich number) wird gepr\"uft. Nicht korrekte Eingaben
		ziehen eine neue Eingabe nach sich. \\ \hline
		
		Spielzug \texttt{build} & build: city
		& Die Syntax und die Semantik der Eingabe (St\"ad existiert, St\"adt hat nicht schon ein
		Hotel) wird gepr\"uft. Nicht korrekte Eingaben
		ziehen eine neue Eingabe nach sich. \\ \hline
		
		Spielzug \texttt{hire} & hire: city
		& Die Syntax und die Semantik der Eingabe (St\"ad existiert, Anzahl der Spieltage reicht noch
		aus, um Zug auszuf\"uhren) wird gepr\"uft. Nicht korrekte Eingaben
		ziehen eine neue Eingabe nach sich. Eine automatische Berechnung erfolgt in den
		\emph{gesperrten} Tagen. \\ \hline
		
		Berechnung des Tagesgewinns &
		& Am Ende eines Spieltages wird der Tagesgewinn berechnet und angezeigt. \\ \hline
		
		Anzeige des Spielstatus am Ende eines Tages &
		& Am Ende eines Tages wird eine Statistik mit der Anzahl der Manager, Hotels, den
		den potenziellen und dem aktuellen Gewinn je Stadt angezeigt. \\ \hline
		
		Pr\"ufung der Syntax f\"ur Befehle &
		& F\"ur jeden definierten Befehl wird die Syntax der Eingabe gepr\"uft und abgeglichen.
		Nicht konforme Angaben werden nicht akzeptiert. Dies schlie\ss{}t insbesondere Sonderzeichen
		ein. \\ \hline
		
		Aufruf der Hilfe & help!
		& Jederzeit kann der Spieler eine Hilfe anzeigen lassen. \\ \hline
	
		Beginn eines neuen Spiels & new game!
		& Hiermit kann der Spieler jederzeit eine neue Runde beginnen. \\ \hline
		
		Spielabbruch auf Wunsch des Spielers & quit!
		& Hiermit kann der Spieler jederzeit das Spiel beenden. \\ \hline
		
		Berechnung des Gesamtgewinns &
		& Am Ende eines Spieltages wird der bis dahin erzielte Gewinn ausgegeben. \\ \hline
		
		Pr\"ufung auf Highscore &
		& Am Ende des Spiels wird zusammen mit dem Namen des Spielers gepr\"uft,
		ob die erreichte Punktzahl unter den 10 besten bisherigen Ergebnissen ist. \"Uber
		das Ergebnis wird der Spieler informiert. \\ \hline

		Darstellung der Highscores &
		& Am Ende des Spiels wird die Highscore-Liste angezeigt.\\ \hline
				
		Abspeichern der Highscore-Liste &
		& Am Ende des Spiels wird der Name des Spielers abgefragt, welcher dann mit der
		erreichten Punktzahl in die High-Score-Datein eingef\"ugt wird, falls die erreichte
		Punktzahl hoch genug ist. \\ \hline
		
		Regul\"ares Spielende &
		& Nachdem die Highscore Liste dargestellt wurde, verabschiedet sich das Spiel vom Spieler.
		\\ \hline
		
	\end{tabular}
\end{center}

\end{exercise}

%\begin{thebibliography}{99}
%\bibitem{Test}
%Hallo
%\end{thebibliography}

%\begin{center}
%\includegraphics[width=6cm,angle=270]{chart.eps}
%\end{center}

\end{document}
