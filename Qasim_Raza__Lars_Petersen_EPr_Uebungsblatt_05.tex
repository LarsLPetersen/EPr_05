\documentclass[a4paper]{article}

%%%%%%%%%%%%%%%%%%%%%%%%%%%%%%%%%%%%%%%%%%%%%%%%%%%%%%%%%%%%%%%%%%%%%%%%%%%%
% Some common includes. Add additional includes you need.
%%%%%%%%%%%%%%%%%%%%%%%%%%%%%%%%%%%%%%%%%%%%%%%%%%%%%%%%%%%%%%%%%%%%%%%%%%%%
\RequirePackage{ngerman}
\RequirePackage[utf8]{inputenc}
\RequirePackage[T1]{fontenc}
\RequirePackage[margin=23mm,bottom=30mm]{geometry}
\RequirePackage{graphicx}
\RequirePackage{amsmath,amsfonts,amssymb,amsthm}
\RequirePackage{hyperref}

%%%%%%%%%%%%%%%%%%%%%%%%%%%%%%%%%%%%%%%%%%%%%%%%%%%%%%%%%%%%%%%%%%%%%%%%%%%%
% Defines for mathematical notation. Add additional defines as needed.
%%%%%%%%%%%%%%%%%%%%%%%%%%%%%%%%%%%%%%%%%%%%%%%%%%%%%%%%%%%%%%%%%%%%%%%%%%%%
\def\O{\mathcal{O}}
\def\sort{\mathrm{sort}}
\def\scan{\mathrm{scan}}
\def\dist{\mathrm{dist}}

\setlength{\parindent}{0cm}
\renewcommand{\refname}{Quellen}
%%%%%%%%%%%%%%%%%%%%%%%%%%%%%%%%%%%%%%%%%%%%%%%%%%%%%%%%%%%%%%%%%%%%%%%%%%%%
% Definition of the assignment header
%%%%%%%%%%%%%%%%%%%%%%%%%%%%%%%%%%%%%%%%%%%%%%%%%%%%%%%%%%%%%%%%%%%%%%%%%%%%
\input{/Users/larspetersen/JWGU/Prg/header.tex}
%%%%%%%%%%%%%%%%%%%%%%%%%%%%%%%%%%%%%%%%%%%%%%%%%%%%%%%%%%%%%%%%%%%%%%%%%%%%

% Set option "german" or "english", depending on what language the
% default texts should be in.
\ExecuteOptions{german}
\ProcessOptions

% Enter the lecture name and semester
\lecture{Einf\"uhrung in die Programmierung}
\semester{Winter 2016/2017}

% Enter your data: Name, Matrikelnummer (student ID number) and group
\student{Qasim Raza, Lars Petersen}{6360278, 6290157}{11}
% Tutorin: sabrinasafre@gmail.com

% Which assignment is this?
\assignment{5}

% The environment "exercise" takes one parameter (the exercise number). 
% This way you can skip exercises if you like. Example:
% 
% \assignment{3}
% \begin{exercise}{8}
% ...
% \end{exercise}
% 
% The solution to exercise 3.8 (3rd assignment, 8th exercise) goes where 
% the dots are.


\begin{document}



\begin{exercise}{x}

Dokumentation der Testf\"alle:


\begin{center}

	{\Large Interne Funktionalit\"at} \\[2ex]

	\begin{tabular}{| p{2.5cm} | p{2.2cm} | p{10cm} |}
		\hline
		Funktionalit\"at & Eingabe & Verhalten des Programms\\ \hline \hline
		
		Begr\"u\ss{}ung & 
		& \dots \\ \hline
		
		Neue Eintr\"age \newline anlegen & notwenidge Attribute
		& \dots \\ \hline
		
		Neue Eintr\"age \newline anlegen & optionale Attribute
		& \dots \\ \hline
		
		Telefonbuch \newline vollst\"andig \newline anzeigen & 
		& Aus einer Liste werden zuf\"allig zwischen 5 und 20 St\"adte ausgew\"ahlt. \\ \hline
		
		Eintr\"age suchen & Angabe einzelner oder mehrerer Attribute
		& \dots \\ \hline
		
		Eintr\"age sortieren und anzeigen & anhand aller Attribute
		& Ber\"ucksichtigung von Sekund\"ar- und Terti\"arkriterien \dots \\ \hline
		
		Eintr\"age bearbeiten & individuell je Attribut
		& \dots \\ \hline
		
		Eintr\"age l\"oschen & 
		& \dots \\ \hline
		
		Telefonbuch neu anlegen und in Datei ablegen & 
		& \dots \\ \hline
		
		Bestehendes Telefonbuch aus Datei einlesen und \"offnen & 
		& \dots \\ \hline
		
		\"Anderungen eines Telefonbuchs abspeichern & 
		& \dots \\ \hline
		
		Test-Telefonbuch &
		& Test-Telefonbuch mit Namen Test-Telefonbuch wurde angelegt. \\ \hline
	\end{tabular}
	
	\newpage
	
	{\Large User Interface} \\[2ex]
		
	\begin{tabular}{| p{2.5cm} | p{2.2cm} | p{10cm} |}
		\hline
		Funktionalit\"at & Eingabe & Verhalten des Programms\\ \hline \hline
				
		[UI] Funktionen navigieren & anlegen, ausgeben, suchen, sortieren, bearbeiten, l\"oschen
		& \dots \\ \hline
		
		[UI] Anlegen neuer Eintr\"age & notwendige Attribute
		& \dots \\ \hline
		
		[UI] Anlegen neuer Eintr\"age & optionale Attribute
		& \dots \\ \hline
		
		[UI] Hilfe aufrufen & 
		&  \\ \hline
		
		[UI] \"Offnen, Bearbeiten und Speichern von Telefonb\"uchern & 
		& \\ \hline
	
		Robustheit: Abfangen \emph{falscher} Eingaben &
		& \dots\\ \hline
				
		
	\end{tabular}
\end{center}

\end{exercise}

%\begin{thebibliography}{99}
%\bibitem{Test}
%Hallo
%\end{thebibliography}

%\begin{center}
%\includegraphics[width=6cm,angle=270]{chart.eps}
%\end{center}

\end{document}
